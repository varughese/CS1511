\documentclass[12pt]{article}
 
\usepackage[margin=1in]{geometry} 
\usepackage{amsmath,amsthm,amssymb, pgfplots}
\usepackage{enumitem} 

\newcommand{\problem}[1]{
	\vskip 1em
	{\large \textbf{#1}}
}
 
\begin{document}
 
\title{CS 1511 Homework 5}%replace X with the appropriate number
\author{Mathew Varughese, Justin Kramer, Zach Smith} %if necessary, replace with your course title
\date{Wednesday, Jan 30}
 
\maketitle

\setlength{\parskip}{.2em}
\setlength\parindent{0pt}
 
\problem{9.}

If a C program can output $x$ and another C program can output $y$, a program can 
be made which calls both of these programs in order. Effectively, this would output
$xy$. Here, $c$ would
be the bits required to create this program that calls the two other programs.

\problem{10.}
Imagine a tree where each node has an infinite number of children. It is still possible to recursively
enumerate over all of these. 

Picture a tree as follows:

{\scriptsize

\hspace{4em} A \hspace{17em} B \hspace{18em} C \ldots

\hspace{1.6em} D \hspace{.9em} E \hspace{1em} F \ldots \hspace{11em} 
G \hspace{1em} H \hspace{1em} I \ldots \hspace{12.3em} 
J \hspace{1em} K \hspace{1em} L \ldots

M \hspace{.01em} N \hspace{.01em} O \ldots \hspace{.2em} 
P \hspace{.01em} Q \hspace{.01em} R \ldots \hspace{.2em} 
S \hspace{.01em} T \hspace{.01em} U \ldots

}

Here, A B C make up the first level and the first level is infinitely long. D E F are children of
A and A has infinite children. G H I are children of B and B has infinite children. J K L are children of 
C and C has infinite children. M N O are children of D and D has infinite children. P Q R are children of E,
etc. M N O P Q R S T U and so on all also have infinite children. 

The idea behind counting them is similar to the proof for prooving that the rational numbers are
countable. Although the above tree is hard to visualize, it will be sufficient for the explanation.

The pattern is as follows: Start at A. Then next, go to B. Next, go to A's first child, D. Then go to 
M. Then go to C. Then go to G. Then go to E. Then go to P. Then go to N. Then go to Ms child. Then go to
sibling of C. Then go to the first child of C (J). Then, now you are on the second level, so 
go back to H and E. Now, go to G's first child. Then that first child. Then go back and do the second child
of everything that was touched. 


It makes more sense to visualize our explanation with a relation to counting $\mathbb{N} \times \mathbb{N}$. 

Drew this out on the back.

\pagebreak


\begin{tabular}{ |c|c|c|c|c| } 
	\hline
	(1,1) & (1,2) & (1,3) & (1,4) & (1,5) \\ 
	(2,1) & (2,2) & (2,3) & (2,4) & (2,5) \\ 
	(3,1) & (3,2) & (3,3) & (3,4) & (3,5) \\ 
	(4,1) & (4,2) & (4,3) & (4,4) & (4,5) \\ 
	(5,1) & (5,2) & (5,3) & (5,4) & (5,5) \\ 
	\hline
\end{tabular}

\vskip 3cm

In the table, each cell represents the all of the children of the parent above. So, say the first 
cell (first row first column) represents A (the first node in the tree). The first
column second row represents \textbf{all} of the possible children of A's children on the second level. 
Every time we "visit" a cell, we go retrace the path and go back through all of the previously visited cells 
and visit the next node.  

So, in the above tree example the order would be: A, B, D, M, \textit{E}, G, \textit{N}, \textit{F}, C, \textit{H}, \textit{O}, \textit{Fs sibling} \ldots 

The italic text represents the nodes visited when "tracing back".

\end{document}