\documentclass[12pt]{article}
 
\usepackage[margin=1in]{geometry} 
\usepackage{amsmath,amsthm,amssymb}
\usepackage{enumitem} 
 
\newcommand{\N}{\mathbb{N}}
\newcommand{\Z}{\mathbb{Z}}
 
\newenvironment{theorem}[2][Theorem]{\begin{trivlist}
\item[\hskip \labelsep {\bfseries #1}\hskip \labelsep {\bfseries #2.}]}{\end{trivlist}}
\newenvironment{lemma}[2][Lemma]{\begin{trivlist}
\item[\hskip \labelsep {\bfseries #1}\hskip \labelsep {\bfseries #2.}]}{\end{trivlist}}
\newenvironment{exercise}[2][Exercise]{\begin{trivlist}
\item[\hskip \labelsep {\bfseries #1}\hskip \labelsep {\bfseries #2.}]}{\end{trivlist}}
\newenvironment{problem}[2][Problem]{\begin{trivlist}
\item[\hskip \labelsep {\bfseries #1}\hskip \labelsep {\bfseries #2.}]}{\end{trivlist}}
\newenvironment{question}[2][Question]{\begin{trivlist}
\item[\hskip \labelsep {\bfseries #1}\hskip \labelsep {\bfseries #2.}]}{\end{trivlist}}
\newenvironment{corollary}[2][Corollary]{\begin{trivlist}
\item[\hskip \labelsep {\bfseries #1}\hskip \labelsep {\bfseries #2.}]}{\end{trivlist}}
 
\begin{document}
 
% --------------------------------------------------------------
%                         Start here
% --------------------------------------------------------------
 
\title{CS 1511 Homework 4}%replace X with the appropriate number
\author{Mathew Varughese, Justin Kramer, Zach Smith} %if necessary, replace with your course title
\date{Monday, Jan 28}
 
\maketitle

\setlength{\parskip}{.2em}
\setlength\parindent{0pt}
 

 \vskip 1em
{\large \textbf{7}}

Assume the definition in (b) is true. Then for a recursively enumerable language L, there exists a Turing machine M with a read/write tape that is initially empty and a write-nly output tape, such that only elements of L are written to the output tape, and every element of L is eventually written to the output tape. 

Now construct a turing machine $M'$.


$M' = $ "On input w:
\begin{enumerate}[topsep=0pt,itemsep=0pt]
\item Run M until it produces a new output on its output tape.
\item Check if w was the item written onto the output tape. If yes, \underline{accept}.
\item Otherwise, go to step 1. If M is halted, \underline{loop indefinitely}.
\end{enumerate}

Machine $M'$ is the same machine defined in part (a). If $x \in L$, then $x$ will show up on the output tape of machine $M$ and will accept. If $x \not\in L$, $M$ will loop indefinitely on $x$.

\

\vskip 1em
{\large \textbf{8. b) }}

The number of bits to represent any string with an equal amount of 0's and 1's will be $2^n$. The probability of finding an incompressible string with this property is $1/\sqrt{n}$. With this in mind, the K(x) Kolmogorov complexity will be greater than or equal to the length of the string if it's incompressible. In this case, the formula will look like this.

\vskip 1em
$n - 1/2\log_2n + c > $ string length
\vskip 1em

As the size of n grows, eventually $1/2\log_2n$ will grow larger than the constant c, and the equation will eventually no longer hold. Thus, there is a finite range that n can exist within where the string will be incompressible. Therefore, there are a finite amount of strings that are incompressible and have this property.



\end{document}