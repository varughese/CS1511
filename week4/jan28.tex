\documentclass[12pt]{article}
 
\usepackage[margin=1in]{geometry} 
\usepackage{amsmath,amsthm,amssymb}
\usepackage{enumitem} 
 
\newcommand{\N}{\mathbb{N}}
\newcommand{\Z}{\mathbb{Z}}
 
\newenvironment{theorem}[2][Theorem]{\begin{trivlist}
\item[\hskip \labelsep {\bfseries #1}\hskip \labelsep {\bfseries #2.}]}{\end{trivlist}}
\newenvironment{lemma}[2][Lemma]{\begin{trivlist}
\item[\hskip \labelsep {\bfseries #1}\hskip \labelsep {\bfseries #2.}]}{\end{trivlist}}
\newenvironment{exercise}[2][Exercise]{\begin{trivlist}
\item[\hskip \labelsep {\bfseries #1}\hskip \labelsep {\bfseries #2.}]}{\end{trivlist}}
\newenvironment{problem}[2][Problem]{\begin{trivlist}
\item[\hskip \labelsep {\bfseries #1}\hskip \labelsep {\bfseries #2.}]}{\end{trivlist}}
\newenvironment{question}[2][Question]{\begin{trivlist}
\item[\hskip \labelsep {\bfseries #1}\hskip \labelsep {\bfseries #2.}]}{\end{trivlist}}
\newenvironment{corollary}[2][Corollary]{\begin{trivlist}
\item[\hskip \labelsep {\bfseries #1}\hskip \labelsep {\bfseries #2.}]}{\end{trivlist}}
 
\begin{document}
 
% --------------------------------------------------------------
%                         Start here
% --------------------------------------------------------------
 
\title{CS 1511 Homework 4}%replace X with the appropriate number
\author{Mathew Varughese, Justin Kramer, Zach Smith} %if necessary, replace with your course title
\date{Monday, Jan 28}
 
\maketitle

\setlength{\parskip}{.2em}
\setlength\parindent{0pt}
 

 \vskip 1em
{\large \textbf{7}}

Assume the definition in (b) is true. Then for a recursively enumerable language L, there exists a Turing machine M with a read/write tape that is initially empty and a write-nly output tape, such that only elements of L are written to the output tape, and every element of L is eventually written to the output tape. 

Now construct a turing machine $M'$.


$M' = $ "On input w:
\begin{enumerate}[topsep=0pt,itemsep=0pt]
\item Run M until it produces a new output on its output tape.
\item Check if w was the item written onto the output tape. If yes, \underline{accept}.
\item Otherwise, go to step 1. If M is halted, \underline{loop indefinitely}.
\end{enumerate}

Machine $M'$ is the same machine defined in part (a). If $x \in L$, then $x$ will show up on the output tape of machine $M$ and will accept. If $x \not\in L$, $M$ will loop indefinitely on $x$.

 \vskip 1em
{\large \textbf{8 (a) }}

 \vskip 1em
{\large \textbf{8 (b) }}

 \vskip 1em
{\large \textbf{8 (c) }}
Assume that the set of incompressible strings contains a infinite subset that is recursively enumerable. This means there is a mapping from the Natural Numbers this infinite subset.  For example, 1 maps to $w_1$, 2 maps to $w_2$, 3 maps to $w_3$, etc. Then construct a Turing Machine M that outputs each of these strings. The subset is infinite, so $\exists x$ such that $|x| >  | <M> | + a + c $. $x$ where a is the natural number that $x$ corresponds to and c is a constant. Then create $M'$ such that $M'$ outputs $x$ on input  $a$. $|<M'>|$ is less than $|x|$ because $M'$ was constructed so that it is of similar length to $M$. This is a contradiction because $x$ is incompressible so no program shorter than it should be able to output it. 

 \vskip 1em
{\large \textbf{8 (d) }}
A set is recursively enumerable if $\exist$ a Turing Machine that outputs each element in the 

\end{document}