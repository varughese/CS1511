\documentclass[12pt]{article}
 
\usepackage[margin=1in]{geometry} 
\usepackage{amsmath,amsthm,amssymb}
\usepackage{enumitem} 

\newcommand{\problem}[1]{
	\vskip 1em
	{\large \textbf{#1}}
}
 
\begin{document}
 
\title{CS 1511 Homework 5}%replace X with the appropriate number
\author{Mathew Varughese, Justin Kramer, Zach Smith} %if necessary, replace with your course title
\date{Wednesday, Jan 30}
 
\maketitle

\setlength{\parskip}{.2em}
\setlength\parindent{0pt}
 
\problem{11 a.}

With an input I = 101 for a Turing machine M, here is one valid computation history H.
In this computation history H, space = *, q\_0 = q\_y, q\_1 = q\_p .

\#q\_y101*\#1q\_p01*\#11q\_p1*\#111q\_p*\#111q\_h*

This configuration thus ends in the halting state.


\setlength{\parskip}{.2em}
\setlength\parindent{0pt}
 
\problem{11 b.}

To begin, here are some defined macros for this problem.

BASE = 7

BASE = 7 is true in this problem because the sum of the number of states, alphabet size, and \# are equal to 7.

PLACE(j) = (H div(base)$^{i+1}$ mod(base)$^i$)

In this case, H is a number that exists which one can interpret as a computation history of M on I.

SAME(i, j, k, l) = (H div(base)$^j$ mod(base)$^i$) =  (H div(base)$^l$ mod(base)$^k$)


\end{document}