\documentclass[12pt]{article}
 
\usepackage[margin=1in]{geometry} 
\usepackage{amsmath,amsthm,amssymb}
\usepackage{enumitem} 

\newcommand{\problem}[1]{
	\vskip 1em
	{\large \textbf{#1}}
}
 
\begin{document}
 
\title{CS 1511 Homework 6}%replace X with the appropriate number
\author{Mathew Varughese, Justin Kramer, Zach Smith} %if necessary, replace with your course title
\date{Friday, Feb 1}
 
\maketitle

\setlength{\parskip}{.2em}
\setlength\parindent{0pt}
 
\problem{11 a.}

With an input I = 101 for a Turing machine M, here is one valid computation history H.
In this computation history H, space = *, q\_0 = q\_y, q\_1 = q\_p .

\#q\_y101*\#1q\_p01*\#11q\_p1*\#111q\_p*\#111q\_h*

This configuration thus ends in the halting state.


\setlength{\parskip}{.2em}
\setlength\parindent{0pt}
 
\problem{11 b.}

To begin, here are some defined macros for this problem.

space = *, q\_0 = q\_y, q\_1 = q\_p

BASE = 7

BASE = 7 is true in this problem because the sum of the number of states, alphabet size, and \# are equal to 7.

PLACE(j) = (H div(BASE)$^{i+1}$ mod(BASE)$^i$)

In this case, H is a number that exists which one can interpret as a computation history of M on I.

SAME(i, j) = (PLACE(i) = PLACE(j))

STATE(i) = (PLACE(i) = q\_y $\lor$ PLACE(i) = q\_p $\lor$ PLACE(i) = q\_h)

TABLE(i,j) = (STATE(i+1), PLACE(i+2) = q\_p0 $\land$ PLACE(j+1), STATE(j+2) = 0q\_p $\land$ SAME(i, j)) $\lor$  (STATE(i+1), PLACE(i+2) = q\_p1 $\land$ PLACE(j+1), STATE(j+2) = 1q\_p $\land$ SAME(i, j)) $\lor$  (STATE(i+1), PLACE(i+2) = q\_p* $\land$ STATE(j+1), PLACE(j+2) = q\_p* $\land$ SAME(i, j))  $\lor$  (STATE(i+1), PLACE(i+2) = q\_y1 $\land$ PLACE(j+1), STATE(j+2) = 0q\_y $\land$ SAME(i, j)) $\lor$  (STATE(i+1), PLACE(i+2) = q\_y0 $\land$ PLACE(j+1), STATE(j+2) = 1q\_y $\land$ SAME(i, j)) $\lor$  (STATE(i+1), PLACE(i+2) = q\_y* $\land$ PLACE(j+1), STATE(j+2) = 0q\_h* $\land$ SAME(i, j))

Godel Sentence S= 

PLACE(i) = \# $\land$ PLACE(j) = \# $\land$ PLACE(k) = \# $\land$ PLACE(l) = \# $\land$

$\forall$x $i < x < j$   $\implies$ PLACE(x) $\neq$ \# $\land$

$\forall$x $j < x < k$   $\implies$ PLACE(x) $\neq$ \# $\land$

$\exists$a $i < i+a < j$   $\implies$ PLACE(i+a) = q\_p $\lor$  PLACE(i+a) = q\_y $\lor$  PLACE(i+a) = q\_h $\land$

$\forall$x $1 \leq x < a-1$   $\implies$ PLACE(i+x) = PLACE(j+x) $\land$

$\forall$x $a+1 < x < j-1$   $\implies$ PLACE(i+x) = PLACE(j+x) $\land$

TABLE(i+a, j+a)

\end{document}