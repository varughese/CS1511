
\documentclass[12pt]{article}
 
\usepackage[margin=1in]{geometry} 
\usepackage{amsmath,amsthm,amssymb}
\usepackage{enumitem}

\newcommand{\problem}[1]{
	\vskip 1em
	{\large \textbf{#1}}
}
 
\begin{document}

\title{CS 1511 Homework 13} % Replace X with the appropriate number
\author{Mathew Varughese, Justin Kramer, Zach Smith} 
\date{Monday, March 4}

\maketitle


\setlength{\parskip}{.2em}
\setlength\parindent{0pt}
 
\problem{24.)}


\problem{25.)}
Take a language L in PH.

L = $\{$ $<$x, k$> \mid$ where $\exists$ a $C_{\mid x \mid}$ Boolean circuit with ${\mid x \mid}^k$ gates, and $\forall$ circuits D with
less than ${\mid x \mid}^k$ gates, the circuit does not compute the same Boolean function as $C_{\mid x \mid}$.

This language is clearly in $\Sigma_2^P$, so therefore L is a language in PH.

A machine to check if a set of strings x is in L will have a runtime of O($2^{x^k}$) when simulated on a Turing Machine.

The Turing Machine will take have to construct every possible circuit with less than ${\mid x \mid}^k$ gates, which will take
O($2^{x^k}$).

With this in mind, the amount of space necessary (circuit complexity necessary) will be $\Omega$($n^k$).

This is due to our circuit being able to hardwire in all the possibilities from our Turing Machine in polynomial time.

In exponential time, we can use poly-space to model our Turing Machine.

Thus, for every k $>$ 0 there is a language in PH whose circuit complexity is $\Omega$($n^k$).


\end{document}
