\documentclass[12pt]{article}
 
\usepackage[margin=1in]{geometry} 
\usepackage{amsmath,amsthm,amssymb}
\usepackage{enumitem}

\newcommand{\problem}[1]{
	\vskip 1em
	{\large \textbf{#1}}
}
 
\begin{document}

\title{CS 1511 Homework 14} % Replace X with the appropriate number
\author{Mathew Varughese, Justin Kramer, Zach Smith} 
\date{Wednesday, March 6}

\maketitle


\setlength{\parskip}{.2em}
\setlength\parindent{0pt}
 

\problem{26. a)}
If a language is in ZPP, then the polynomial-time PTM M can prove that
the language is in ZPP. 

For this Turing Machine, it will either output the correct answer or "DON'T KNOW".

There is at most a 1/2 chance that it outputs "DON'T KNOW". If the machine is run once,
then the chance that it outputs this is at most 1/2, with at least a 1/2 chance of outputting
the correct answer.

If the machine is run twice, there is at most a 1/4 chance that it doesn't output the correct answer.

If you take this to a limit of infinity, there is no chance that you do not eventually output the correct answer
when you run the machine an infinite amount of times. It will take poly-time to run, so it doesn't matter if we simulate
this machine for many many times.

In the other direction, we can use Markov's inequality to show that with our PTM M, L is in ZPP.

With this inequality, we have that the P(X $\geq$ a) $\leq E(X)/a$.

In this case, we can make a = 3E(X) for an example. Therefore, the probability that our X, our runtime, is greater
than expected poly-time is now 1/3. 

If we take this to a limit of inifinity, the probability reduces to 0 that our expected runtime is greater than polynomial time.

We can take our PTM M and run it to build our language L. This language will be in ZPP because if the PTM outputs an answer
that isn't "DON'T KNOW", then it must be a correct answer. This is necessary for ZPP. We can count these inputs as part 
of our language L.

With a runtime established and correct answers established to build a language L, then we have shown that if our PTM M
exists, we can build a language L in ZPP.


\problem{26. b)}

In order to show that $ZPP = RP \cap coRP$, we need to show two things.

We will begin by showing that $ZPP \subseteq RP \cap coRP$.

To start this, we must show that $ZPP \subseteq RP$ and $ZPP \subseteq coRP$.

We will begin by showing that $ZPP \subseteq coRP$.

Let L $\in$ ZPP. Then there exists a PTM M that for all x either correctly decides 
$x \in L$ or outputs "Don't know". 

Let M' be a TM that on all input x, returns 1 if M(x) = "Don't know" and otherwise returns M(x).

So if $x \in L$, M(x) = 1 or M(x) = "Don't know" so M'(x) = 1
if $x \notin L$ M(x) = 0 correctly or M(x) = "Don't know", and here M'(x) returns incorrect
with probability $\leq$ 1/2.

Therefore, this clearly demonstrates that $ZPP \subseteq coRP$.

A very similar proof demonstrates that $ZPP \subseteq RP$ with the same methodology, except
that M' returns 0 if M(x) = "Don't know".

Now we must demonstrate the second part, which is that $RP \cap coRP \subseteq ZPP$.

To show this, we begin with a language L $\in RP \cap coRP$. There exists two TM's from this, with

$M_1$(x) = 1 if $x\in$ L, and is incorrect for $x\notin$ L with probability $\leq$ 1/2.

$M_2$(x) = 0 if $x\notin$ L, and is incorrect for $x\in$ L with probability $\leq$ 1/2.

We then construct a PTM M' which works as follows:

Run $M_1$(x) and if $M_1$(x) = 0, return x $\notin$ L.

Run $M_2$(x) and if $M_2$(x) = 1, return x $\in$ L.

Else return "Don't know"

In this way, if M'(x) returns a 0 or a 1, it is correct.

If M'(x) returns "Don't know" with probability $\leq$ 1/2 of being correct.

This is therefore ZPP, and we have shown that $RP \cap coRP \subseteq ZPP$.

With both of these proven, this shows that $ZPP = RP \cap coRP$.

\problem{27}
Lemma 7.12 says that 
A coin with Pr[Heads] = p
can be simulated by a PTM in 
expected time O(1) provided the 
ith bit of p is computable in 
poly(i) time.

We can construct this p such that 
a random coin that comes up with this probability
will make a Turing Machine able to 
decide a language in undecidable time.

Assume that this is possible to make a PTM 
that simulates a coin flip with probability p. The ith
bit of p can be computed in constant time.

The p is comprised of 1s and 0s. The turing machine
in question (say TM T) generates a random "x". This 
x is a random variable. Then we compare x to that
probability p. We do this by comparing bit
by bit. If p is 0.0101010 and x is 0.0101011, 
we know that x has a greater probability than 
p.

We make each bit of this p represent whether a 
Turing Machine will halt on a particular language.
So, order every Turing Machine and input w
like (M,w) and correspond it with a integer. 
The ith integer in p will be a 1 if M halts on w
and 0 if not.

Then, the Turing Machine T could solve the halting
problem. To figure out if M halts on w,
it would first correspond that M,w pair to an integer.
Call this integer i. The ith digit of p would tell
us what the answer to this question is. So we flip 
We flip a coin i times. We count the 
distribution of heads and tails. We can say with p 
confidence whether or not M halts w.

Therefore, our number p is "0." followed by a series of 1's and 0's that allows
an undecidable language to be decided.


\problem{28}
First we show NP $\in$ PP. Then we show coNP $\in$ PP. This will 
prove that NP $\cup$ coNP $\in$ PP.

If L is in NP, then there exists a TM T
that verifies an answer to L in a polynomial number
of steps.

For every x, x is in L, iff $\exists y$ 
T(x, y) accepts.

So build a $T'$ so it tries w values for y. If this accepts, accept.
If not, flip a coin. The coin should biased so it gives heads
slighter less than half of the time. This "slighter half" will
be of size $w/2^n$.
If the coin returns heads, accept.
Otherwise reject. This will make it so if an x is in L, it will 
accept with more than 1/2 probability. 

So, NP is in PP. 

CoNP is in PP for a similar reason. If L is in CoNP, there exists
a TM T that halts on an answer L in a polynomial number of steps.

For every x, x is in L, iff $\forall y$ T(x, y) accepts.

So, x is not in L if there exists a y that T(x, y) rejects. We apply 
the same logic of the previous problem. We try some inputs for y. If they
cause it to fail, than we reject. Then, we flip a biased coin that gives
heads slightly higher than 1/2 of the time. Then we reject.

Thus, CoNP is in PP and NP $\cup$ coNP $\in$ PP.





\end{document}
