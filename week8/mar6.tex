
\documentclass[12pt]{article}
 
\usepackage[margin=1in]{geometry} 
\usepackage{amsmath,amsthm,amssymb}
\usepackage{enumitem}

\newcommand{\problem}[1]{
	\vskip 1em
	{\large \textbf{#1}}
}
 
\begin{document}

\title{CS 1511 Homework 14} % Replace X with the appropriate number
\author{Mathew Varughese, Justin Kramer, Zach Smith} 
\date{Wednesday, March 6}

\maketitle


\setlength{\parskip}{.2em}
\setlength\parindent{0pt}
 
\problem{26. a)}
If a language is in ZPP, then the polynomial-time PTM M can prove that
the language is in ZPP. 

For this Turing Machine, it will either output the correct answer or "DON'T KNOW".

There is at most a 1/2 chance that it outputs "DON'T KNOW". If the machine is run once,
then the chance that it outputs this is at most 1/2, with at least a 1/2 chance of outputting
the correct answer.

If the machine is run twice, there is at most a 1/4 chance that it doesn't output the correct answer.

If you take this to a limit of infinity, there is no chance that you do not eventually output the correct answer
when you run the machine an infinite amount of times. It will take poly-time to run, so it doesn't matter if we simulate
this machine for many many times.

In the other direction, we can use Markov's inequality to show that with our PTM M, L is in ZPP.



\problem{26. b)}

\problem{27.)}


\problem{28.)}

\end{document}
