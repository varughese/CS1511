\documentclass[12pt]{article}
 
\usepackage[margin=1in]{geometry} 
\usepackage{amsmath,amsthm,amssymb}
\usepackage{enumitem}

\newcommand{\problem}[1]{
	\vskip 1em
	{\large \textbf{#1}}
}
 
\begin{document}

\title{CS 1511 Homework 14} % Replace X with the appropriate number
\author{Mathew Varughese, Justin Kramer, Zach Smith} 
\date{Wednesday, March 6}

\maketitle


\setlength{\parskip}{.2em}
\setlength\parindent{0pt}
 

\problem{26. a)}
If a language is in ZPP, then the polynomial-time PTM M can prove that
the language is in ZPP. 

For this Turing Machine, it will either output the correct answer or "DON'T KNOW".

There is at most a 1/2 chance that it outputs "DON'T KNOW". If the machine is run once,
then the chance that it outputs this is at most 1/2, with at least a 1/2 chance of outputting
the correct answer.

If the machine is run twice, there is at most a 1/4 chance that it doesn't output the correct answer.

If you take this to a limit of infinity, there is no chance that you do not eventually output the correct answer
when you run the machine an infinite amount of times. It will take poly-time to run, so it doesn't matter if we simulate
this machine for many many times.

In the other direction, we can use Markov's inequality to show that with our PTM M, L is in ZPP.



\problem{26. b)}

\problem{27}
Lemma 7.12 says that 
A coin with Pr[Heads] = ρ 
can be simulated by a PTM in 
expected time O(1) provided the 
ith bit of ρ is computable in 
poly(i) time.

We can construct this p such that 
a random coin that comes up with this probability
will make a Turing Machine able to 
decide a language in undecidable time.

Assume that this is possible to make a PTM 
that simulates a coin flip with probability p. The ith
bit of p can be computed in constant time.

The p is comprised of 1s and 0s. The turing machine
in question (say TM T) generates a random "x". This 
x is a random variable. Then we compare x to that
probability p. We do this by comparing bit
by bit. If p is 0.0101010 and x is 0.0101011, 
we know that x has a greater probability than 
p.

We make each bit of this p represent whether a 
Turing Machine will halt on a particular language.
So, order every Turing Machine and input w
like (M,w) and correspond it with a integer. 
The ith integer in p will be a 1 if M halts on w
and 0 if not.

Then, the Turing Machine T could solve the halting
problem. To figure out if M halts on w,
it would first correspond that M,w pair to an integer.
Call this integer i. The ith digit of p would tell
us what the answer to this question is. So we flip 
We flip a coin i times. We count the 
distribution of heads and tails. We can say with p 
confidence whether or not M halts w.


\problem{28}
First we show NP $\in$ PP. Then we show coNP $\in$ PP.

\end{document}
