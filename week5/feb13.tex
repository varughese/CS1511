\documentclass[12pt]{article}
 
\usepackage[margin=1in]{geometry} 
\usepackage{amsmath,amsthm,amssymb, pgfplots}
\usepackage{enumitem} 

\newcommand{\problem}[1]{
	\vskip 1em
	{\large \textbf{#1}}
}
 
\begin{document}

\title{CS 1511 Homework 7} % Replace X with the appropriate number
\author{Mathew Varughese, Justin Kramer, Zach Smith} 
\date{Wednesday, Feb 13}

\maketitle


\setlength{\parskip}{.5em}
\setlength\parindent{0pt}
 
\problem{14.}
First we prove that Definition 1 implies Definition 2. This is straight forward.

Assume Definition 1 holds true. This means there exists a Turing Machine $M$, such
that $M$ accepts $x$ iff $x \in L$. M halts on any $x$ in $ T(|x|)$ steps.

Now, construct N that is identical to M and set $b = 1$. Now the second definition holds true, because
N will halt on x within $1 * T(|x|)$ steps. 

Now we prove the other direction. Assume Definition 2 is true. We have a 
machine N that accepts $x$ if is $x \in L$. It also halts within $b * T(|x|)$ steps.
Now, we will speed up the Turing Machine by a constant factor $b$. This is possible
because we can use a larger tape alphabet. We can speed it up by a factor of 2 by
making the new tape alphabet the characters that are the merge of two characters 
on the tape. This will halven the tape length which would speed up the TM by 
2. This can be done as many times until the TM is b times faster. In other words,
we do this process $b/2$ times. Call this new TM M. Now, Definition 1 holds 
true. 



\end{document}