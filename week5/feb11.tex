\documentclass[12pt]{article}
 
\usepackage[margin=1in]{geometry} 
\usepackage{amsmath,amsthm,amssymb, pgfplots}
\usepackage{enumitem} 

\newcommand{\problem}[1]{
	\vskip 1em
	{\large \textbf{#1}}
}
 
\begin{document}

\title{CS 1511 Homework 6} % Replace X with the appropriate number
\author{Mathew Varughese, Justin Kramer, Zach Smith} 
\date{Monday, Feb 11}

\maketitle


\setlength{\parskip}{.2em}
\setlength\parindent{0pt}
 
\problem{12. (a)}
A finite state machine could be constructed such that its language is the negation
of the language of M. This can be done as it was in 1502. The accept states turn 
into reject states and vice versa. 

\problem{12. (b)}
This can be done by creating a DFA (P) that has a language which the intersection
of the language of L and the language of N. This can be done by making the states of P
be the cross product of the states from L and the states of N. Then, the accept states
are those states that have both accept states from L and N. The transition function
would be determined by going to the state ($l$,$n$). $l$ is what the 
transition function would do from DFA L and $n$ is where the transition function 
for DFA N. The start state would be the the state that contains both of the start states.

\problem{12. (c)}
The states in Q will be the power set of the states of P. Say that the characters 
in the alphabet for P were 4 bits (for x, y, z, w). This can be assumed 
as we said they were properly encoded. Now, the characters in the alphabet will be 3
bits long (x, y, z). The transition function will change so that it will change to the 
state that contains the states that P would go to if w was a 0 or 1. This explanation works
better with an example. Say in a DFA state A goes to B when w=0 and A goes to C when w=1. 
In this new DFA, the set of states will be the power set of $\{ A,B,C \}$. So the states would 
be $\{\{A\}, \{B\}, \{C\}, \{A, B\}, \{A, C\}, \{B, C\}, \{A, B, C\}\}$. Since it is an existential,
we want to consider all possibilities for w. This means that from state A, the next state (given the 
same x, y, and z) would be $\{B, C\}$. This would obviously get more complicated, but it is definitely
possible.

\problem{12. (d)}
This will be done in a similar way to problem c. $\forall x P(x)$ is logically equivalent 
to $\neg \exists \neg P(x)$. So,  $\forall z \exists w \ (x + y = z) \wedge \neg (y + z
= w + x)$ is logically equivalent to $\neg \exists \ z ( \neg \exists w \  (x + y = z) \wedge \neg (y + z
= w + x))$ To construct this, we first create a DFA that has a language which is the opposite of Q.
Then, following the same idea in Part C, create a DFA that has states which are the power set
of Q. Then, take that DFA and construct a new one that contains the complement of that ones's language.  

\problem{12. (e)}
The same idea can be done for machine S.

\problem{12. (f)}
This problem is the same as asking whether $L(S)$ is non empty. The language DFA is empty 
if there are no accept states, or if the accept states are not reachable from
the start state. The algorithm for determining if $L(S)$ is not empty would be as follows: 

\begin{enumerate}[topsep=0pt,itemsep=0pt]
\item Mark the start state of S
\item Repeat until no new states are marked:
\item \hspace{1em} Mark any state that has a transition coming into from a state that is marked
\item If no accept state is marked \underline{reject} otherwise \underline{accept}.
\end{enumerate}

\problem{12. (g)}
Take the first order and break each piece between an and into a DFA of proper encoding.
Then, continually construct each DFA from each "or" or "and" or "negation" by using the 
same techniques that were done in previous problems. Then from the inside out, construct 
new DFAs based on the quantifiers. If it is existential, then use the method in 12 c and if it is 
a universal, then use the method in 12 d. After this finite automata is constructed, call it S and use
the algorithm in part f to determine if it accepts any string. If it does, then it is true. Otherwise 
it is false. 

\problem{13. (a)} 

For this to be in $TIME(n^2)$ it would need to take steps proportional to the square of $|I|$.
Saying $a = 100*(b_I)^2$ doesn't work because it would depend on $b_I$ making it not a constant.
$b_I$ is dependent on $I$, so that would mean $a$ would change.. You need to show that the Turing Machine stops in 
$a|I|^c$ steps, so $a$ needs to be a constant. In this case it is not a constant. 


\problem{13. (b)}

What we need to prove here is that with our Turing Machine S that it will take at most
$dn^22^{2n}$ steps. In this case, our n would be equal to the length our input I. So we are saying
that our Turing Machine S has input I and a string version of our Turing Machine and input
$b_I$ (with some constant c) such that $cI^2(b_I)^2 \leq dn^22^{2n}$. Given a 
string of length $|I|$, the maximum possible number of possibilities would 
be $2^{|I|}$, or $2^n$. This means at maximum $b_I$ can only be $2^n$.
Therefore, after substituting $2^n$ for $b_I$, we get the expression 
$cn^2(2^n)^2 \leq dn^22^{2n}$. This makes it clear that 
$cI^2(b_I)^2 \leq cn^2(2^n)^2 \leq dn^22^{2n}$

\problem{13. (c)}

i.

L' is not in TIME(n) because the language is defined as a collection of strings that
a Turing machine can run and it will not accept within $b|I|$ steps. An attached sheet
will show this proof by diagonalization.

ii.

L' is in $TIME(n^c)$ for, at the smallest, a constant c = 4. This is because in unary,
our encoding of $b_I$ will now be able to store n numbers with n digits. In this case, our
formula for the least amount of steps will be (with a constant d) $dI^2(b_I)^2$. In this case, 
$I^2$ will be $n^2$ and $(b_I)^2$ will also be $n^2$, so the overall algorithm will take at 
minimum $n^4$ steps.




\end{document}