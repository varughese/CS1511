\documentclass[12pt]{article}
 
\usepackage[margin=1in]{geometry} 
\usepackage{amsmath,amsthm,amssymb}
\usepackage{enumitem} 

\newcommand{\problem}[1]{
	\vskip 1em
	{\large \textbf{#1}}
}
 
\begin{document}

\title{CS 1511 Homework 26} % Replace X with the appropriate number
\author{Mathew Varughese, Justin Kramer, Zach Smith} 
\date{Mon, Apr 15}

\maketitle


\setlength{\parskip}{.2em}
\setlength\parindent{0pt}
 
\problem{53.}
Each bit has either a probabiltiy of 1 or 0. There
is not a way to tell how the message 
was formed, so a machine can be made that 
outputs a coin flip that has the same answer.

\problem{54.}
If P=NP then any problem that can be solved by a 
nondeterministic polynomial TM can be solved 
by a deterministic polynomial TM. A one way function 
can be inverted if one tests all possible values $x$
to check if $f(x)=y$ where y is the output that
is trying to be reversed. This is a NP problem, because
the certificate would be x. So, if P=NP, then 
this could be solved in polynomial time, and any
one-way function would be reversible in polynomial
time, thus contradicting the defintion of a one way
function. 

\problem{55.}
The distributions of $E_{U_n}(x)$ and $E_{U_n}(x')$
can not be identical if $n < m$ because there must be
an n that maps to two m's, which means that if you have
one of the n's then there's a greater possibility of these two
answers for your initial message, which destroys the 
possibility that you have the same distribution for all
messages once run through the function E. 

You need to have at least one n for each m so that
for each key n, it could map to any m so that all
distributions of E are identical.

\problem{56.}
Assume by contradiction that $f^k$ is not a one-way
permutation. Now say we have a polynomial time 
algorithm A that we apply $f^k$ to in order to get
a probability for (A(y) = y) that is greater than
some negligible amount for all n.

We can then convert our probability with 
this composition into $f(f^{k-1} (A(y) = y)$.

We know that $f^{k-1} A(y)$ can be computed in 
polynomial time. So we can compute A(y) and then
apply our permutation f (k - 1) times, which will 
takes $n^c$ (a poly amount of times of f which
is poly-time). This now contradicts that f is a 
one-way permutation.  

\end{document}