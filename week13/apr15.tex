\documentclass[12pt]{article}
 
\usepackage[margin=1in]{geometry} 
\usepackage{amsmath,amsthm,amssymb}
\usepackage{enumitem} 

\newcommand{\problem}[1]{
	\vskip 1em
	{\large \textbf{#1}}
}
 
\begin{document}

\title{CS 1511 Homework 26} % Replace X with the appropriate number
\author{Mathew Varughese, Justin Kramer, Zach Smith} 
\date{Mon, Apr 15}

\maketitle


\setlength{\parskip}{.2em}
\setlength\parindent{0pt}
 
\problem{53.}
Each bit has either a probabiltiy of 1 or 0. There
is not a way to tell how the message 
was formed, so a machine can be made that 
outputs a coin flip that has the same answer.

\problem{54.}
If P=NP then any problem that can be solved by a 
nondeterministic polynomial TM can be solved 
by a deterministic polynomial TM. A one way function 
can be inverted if one tests all possible values $x$
to check if $f(x)=y$ where y is the output that
is trying to be reversed. This is a NP problem, because
the certificate would be x. So, if P=NP, then 
this could be solved in polynomial time, and any
one-way function would be reversible in polynomial
time, thus contradicting the defintion of a one way
function. 

\problem{55.}
$f(x)$ is a one way permutation. A one way
permutation is a function that maps n bits to
n bits and for every BPP machine C,
C(f(x)) = x has a very small probability. 

Thus, f(f(x)) will have the same property. Since
x is not discoverable from f(x), f(f(x)) also 
has the same property. It is a one way function.

This idea can be repeated for all $f^k$ functions.
Since $k$ is polynomial, 


\end{document}