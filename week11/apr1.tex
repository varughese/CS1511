\documentclass[12pt]{article}
 
\usepackage[margin=1in]{geometry} 
\usepackage{amsmath,amsthm,amssymb}
\usepackage{enumitem} 

\newcommand{\problem}[1]{
	\vskip 1em
	{\large \textbf{#1}}
}
 
\begin{document}

\title{CS 1511 Homework 20} % Replace X with the appropriate number
\author{Mathew Varughese, Justin Kramer, Zach Smith} 
\date{Monday, April 1st}

\maketitle

\setlength{\parskip}{.2em}
\setlength\parindent{0pt}

\problem{37.}



\problem{38.}
The Toffoli gate takes 3 bits. If the first bit are set 1, it performs a
NOT on the third bit. Otherwise, they stay the same.

It is Universal. 

NOT = 1, 1, A -> performs NOT A.

AND = $I_1$, $I_2$, 0 -> performs an AND on $I_1$, $I_2$

We can make a NAND gate because we take an AND gate and a NOT gate after that.

OR = OR can be built from a NAND gate (NOT AND). Take the two
inputs and put them into two NAND gates. Take the outputs of those NAND
gates and input them into another NAND gate. This will output the
value of an OR gate.


\problem{39 a.}
$
\begin{bmatrix} 
	1/\sqrt{2} & 1/\sqrt{2} \\
	1/\sqrt{2} & -1/\sqrt{2}
\end{bmatrix}
\begin{bmatrix} 
	0 & 1 \\
	1 & 0
\end{bmatrix}
\begin{bmatrix} 
	1/\sqrt{2} & 1/\sqrt{2} \\
	1/\sqrt{2} & -1/\sqrt{2}
\end{bmatrix}
=
\begin{bmatrix} 
	1/\sqrt{2} & 1/\sqrt{2} \\
	1/\sqrt{2} & -1/\sqrt{2}
\end{bmatrix}
\begin{bmatrix} 
	1/\sqrt{2} & -1/\sqrt{2} \\
	1/\sqrt{2} & 1/\sqrt{2}
\end{bmatrix}
=
\begin{bmatrix} 
	1 & 0 \\
	0 & -1
\end{bmatrix}
$

$
\begin{bmatrix} 
	1 & 0 \\
	0 & -1
\end{bmatrix}
\begin{bmatrix} 
	a \\
	-b
\end{bmatrix}
$

\problem{39 b.}
$a^2$ 

\problem{39 c.}
$(-b)^2$

\problem{40 a.}
(10.6 in the text)

When you measure the register, 
you are not changing the state. 
You will output the value with 
the probability defined in v.

\problem{40 b.}
The same logic applies. Measuring
the first qubit will reveal the output
with the probability defined in v.
This does not change the second qubit.

\problem{40 c.}
Same logic applies. Opening the second qubit
does not change the logic of the 
first one.

\problem{41 a.}
$
\begin{bmatrix} 
	1/\sqrt{2} & 1/\sqrt{2} & 0 & 0 \\
	1/\sqrt{2} & -1/\sqrt{2} & 0 & 0 \\
	0 & 0 & 1 & 0 \\
	0 & 0 & 0 & 1 \\
\end{bmatrix}
$

\problem{41 b.}
$
\begin{bmatrix} 
	1 & 0 & 0 & 0 \\
	0 & 1 & 0 & 0 \\
	0 & 0 & 1/\sqrt{2} & 1/\sqrt{2} \\
	0 & 0 & 1/\sqrt{2} & -1/\sqrt{2} \\
\end{bmatrix}
$

\problem{41 c.}
$
\begin{bmatrix} 
	1/\sqrt{2} & 1/\sqrt{2} & 0 & 0 \\
	1/\sqrt{2} & -1/\sqrt{2} & 0 & 0 \\
	0 & 0 & 1/\sqrt{2} & 1/\sqrt{2} \\
	0 & 0 & 1/\sqrt{2} & -1/\sqrt{2} \\
\end{bmatrix}
$

\problem{41 d.}
$
\begin{bmatrix} 
	1/\sqrt{2} & 1/\sqrt{2} & 0 & 0 \\
	1/\sqrt{2} & -1/\sqrt{2} & 0 & 0 \\
	0 & 0 & 1/\sqrt{2} & 1/\sqrt{2} \\
	0 & 0 & 1/\sqrt{2} & -1/\sqrt{2} \\
\end{bmatrix}
*
\begin{bmatrix} 
	a \\
	b \\
	c \\
	d \\
\end{bmatrix}
=
\begin{bmatrix} 
	a/\sqrt{2} + b/\sqrt{2}\\
	a/\sqrt{2} - b/\sqrt{2}\\
	c/\sqrt{2} + d/\sqrt{2}\\
	c/\sqrt{2} - d/\sqrt{2}\\
\end{bmatrix}
$

\problem{41 e.}
$
\begin{bmatrix} 
	1/\sqrt{2} & 1/\sqrt{2} & 0 & 0 \\
	1/\sqrt{2} & -1/\sqrt{2} & 0 & 0 \\
	0 & 0 & 1 & 0 \\
	0 & 0 & 0 & 1 \\
\end{bmatrix}
*
\begin{bmatrix} 
	a \\
	b \\
	c \\
	d \\
\end{bmatrix}
=
\begin{bmatrix} 
	a/\sqrt{2} + b/\sqrt{2}\\
	a/\sqrt{2} - b/\sqrt{2}\\
	c \\
	d \\
\end{bmatrix}
$


\problem{41 f.}
$
\begin{bmatrix} 
	1 & 0 & 0 & 0 \\
	0 & 1 & 0 & 0 \\
	0 & 0 & 1/\sqrt{2} & 1/\sqrt{2} \\
	0 & 0 & 1/\sqrt{2} & -1/\sqrt{2} \\
\end{bmatrix}
*
\begin{bmatrix} 
	a/\sqrt{2} + b/\sqrt{2}\\
	a/\sqrt{2} - b/\sqrt{2}\\
	c \\
	d \\
\end{bmatrix}
=
\begin{bmatrix} 
	a/\sqrt{2} + b/\sqrt{2}\\
	a/\sqrt{2} - b/\sqrt{2}\\
	c/\sqrt{2} + d/\sqrt{2}\\
	c/\sqrt{2} - d/\sqrt{2}\\
\end{bmatrix}
$


\end{document}