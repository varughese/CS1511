\documentclass[12pt]{article}
 
\usepackage[margin=1in]{geometry} 
\usepackage{amsmath,amsthm,amssymb}
\usepackage{enumitem} 

\newcommand{\problem}[1]{
	\vskip 1em
	{\large \textbf{#1}}
}
 
\begin{document}

\title{CS 1511 Homework 21} % Replace X with the appropriate number
\author{Mathew Varughese, Justin Kramer, Zach Smith} 
\date{Wednesday, April 3}

\maketitle

\setlength{\parskip}{.2em}
\setlength\parindent{0pt}

\problem{42 a.}

We begin with the EPR experiment from the book. We give Alice x and Bob y. Both of them see that
their variables equal 1. In this case, they will both decide to rotate their qubits according to the book's
protocol.

The qubits of Alice and Bob begin as $\mid ij \rangle = 1/\sqrt{2}\mid 00 \rangle + 1/\sqrt{2}\mid 11 \rangle$

To start, Alice rotates her qubit by $\pi / 8$.

The matrix representing a rotation of $\pi / 8$ to our qubit is:

$
\begin{bmatrix} 
	cos(\pi / 8) & cos(5\pi / 8) \\
	sin(\pi / 8) &  sin(5\pi / 8)
\end{bmatrix}
$

\vskip .5cm

This matrix comes form taking our original $\pi / 8$ matrix which is : 

$
\begin{bmatrix} 
	cos(\pi / 8) & 0 \\
	sin(\pi / 8) &  0
\end{bmatrix}
$

and multiplying by:

$
\begin{bmatrix} 
	1  \\
	0
\end{bmatrix}
$

to get:

$
\begin{bmatrix} 
	cos(5\pi / 8) \\
	sin(5\pi / 8)
\end{bmatrix}
$

When combined, we get the overall rotation we have above.

After this rotation is applied, our qubit $\mid ij \rangle = 1/\sqrt{2} * cos(\pi / 8)\mid 00 \rangle + 1/\sqrt{2} * sin(\pi / 8)\mid 10 \rangle + 1/\sqrt{2} * cos(5\pi / 8)\mid 01 \rangle + 1/\sqrt{2} * sin(5\pi / 8)\mid 11 \rangle$

Now Bob decides to rotate by $-\pi / 8$ since he sees $y=1$, and the protocol says that causes this rotation.

By the same process as above, we get the rotation of $-\pi / 8$ to be:

$
\begin{bmatrix} 
	cos(-\pi / 8) & cos(-5\pi / 8) \\
	sin(-\pi / 8) &  sin(-5\pi / 8)
\end{bmatrix}
$

\vskip .5cm

Now we apply this rotation to our already rotated qubit, making our qubit $\mid ij \rangle = (1/\sqrt{2} * cos(\pi / 8)\mid 0 \rangle + 1/\sqrt{2} * sin(\pi / 8)\mid 1 \rangle)(1/\sqrt{2} * cos(\pi / 8)\mid 0 \rangle - 1/\sqrt{2} * sin(\pi / 8)\mid 1 \rangle) + (1/\sqrt{2} * - sin(\pi / 8)\mid 0 \rangle + 1/\sqrt{2} * cos(\pi / 8)\mid 1 \rangle)(1/\sqrt{2} * sin(\pi / 8)\mid 0 \rangle + 1/\sqrt{2} * sin(\pi / 8)\mid 1 \rangle)$

Now we do the multiplications to simplify. Since $1/\sqrt{2}$ is in all parts, we can simplify it out of our math.

After simplifying, we get the following state of our qubit.

$\mid ij \rangle = (cos^2(\pi/8) - sin^2(\pi/8))\mid 00 \rangle - 2 sin(\pi/8)cos(\pi/8)\mid 01 \rangle + 2sin(\pi/8)cos(\pi/8)\mid 10 \rangle + (cos^2(\pi/8) - sin^2(\pi/8))\mid 11 \rangle$

Now we just have the measuring of Alice and Bob. To start, let's find out what we need to win the game.

In order to win the game with x = 1 and y = 1, we will need a and b to be different. Thus, Alice and Bob must give different answers.

We can calculate the chance of each possible path down the possibilities tree. We start with the probability that we get $\mid00\rangle$, which is

equal to the probability of $\mid11\rangle$. This probability is the square of the amplitude, which is $(cos^2(\pi/8) - sin^2(\pi/8))^2$. This is equal to .5.

Also, the probability of getting $\mid01\rangle$ or $\mid10\rangle$ are the same, and the squares of their amplitudes are $(2sin(\pi/8)cos(\pi/8))^2$, which is also .5.

Since everything has equal probabilities, the probability of having different values for a and b are .5. 

This means that the chance of Alice and Bob winning in this case is 1/2.


\problem{42 b.}
This probability is the same as the probability of
Alice and Bob winning for the protocol
in the textbook.

There are 3 cases:

1. If $x = y = 0 $, $a = b$ w probability 1

2. If $x \neq y $ then $a = b$ with probability $cos^2(\pi/8) (> 85) $

3. If $x = y = 1$, then $a = b$ with probability $1/2$.

The calculations are the same because 
the order Bob and Alice look at their 
qubits does not matter.

The overall acceptance probability is 
at least $1/4 * 1 + 1/2 * 0.85 + 1/4 * 1/2 = 0.8$.

This is the same as the value in the textbook.

Each case can be calculated the 
same as the way was done in class. This is 
written on page 208 in the textbook.
The protocol is the same. Splitting and rotating
individual qubits is possible.


\end{document}