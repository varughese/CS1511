\documentclass[12pt]{article}
 
\usepackage[margin=1in]{geometry} 
\usepackage{amsmath,amsthm,amssymb}
\usepackage{enumitem} 

\newcommand{\problem}[1]{
	\vskip 1em
	{\large \textbf{#1}}
}
 
\begin{document}

\title{CS 1511 Homework 22} % Replace X with the appropriate number
\author{Mathew Varughese, Justin Kramer, Zach Smith} 
\date{Fri, April 5}

\maketitle


\setlength{\parskip}{.2em}
\setlength\parindent{0pt}
 
\problem{43.}

\problem{44 a.}
Alice can perform rotations 
that are multiples of $\pi/4$. Take
the value of x and y combined
and use those to determine
how many degrees to rotate
the qubit by. Say x=0, y=0,
then rotate the qubit by 0 degrees.
x=0, y=1, then rotate by $\pi/4$ degrees.
If x =1 y=0, rotate by $2\pi/4$. If
x = 1, y = 1, rotate by $3\pi/4$ degrees.


\problem{44 b.}
The state of b will be unchanged, 
because Alice performed
rotations only on the first qubit.
The state of a will now be
dependent on the value of x and y.

\problem{44 c.}

If a $= 0^n$, Simon's algorithm still works. This is because if the function is one-to-one, 
and $a = 0^n$, after we compute $\mid xz\rangle -> \mid x(y \oplus f(x))\rangle$ we 
can measure $\mid (x \oplus a)$ and see that it's equivalent to x. This will let us know
that $a = 0^n$. We will therefore have correctly computed a. Or, if we continue Simon's
algorithm, we will eventually be finding k linear equations for $y \odot a = 0$ with a 
uniform string for y that makes this true. In this case, every single one of these y's will 
work. Solving the linear equations will give us that all values of a are 0, which is true.

\problem{44 a.}
To get the Bell state $1/\sqrt{2}\mid0\rangle + 1/\sqrt{2}\mid1\rangle$, Alice can perform a rotation of 
$\pi/4$ to her qubit. This could be when x = 0 and y = 0.

\vskip .3cm

To get the Bell state $1/\sqrt{2}\mid0\rangle - 1/\sqrt{2}\mid1\rangle$, Alice can perform a rotation of 
$-\pi/4$ to her qubit. This could be when x = 0 and y = 1.

\vskip .3cm

To get the Bell state $-1/\sqrt{2}\mid0\rangle + 1/\sqrt{2}\mid1\rangle$, Alice can perform a rotation of 
$3\pi/4$ to her qubit. This could be when x = 1 and y = 0.

\vskip .3cm

To get the Bell state $-1/\sqrt{2}\mid0\rangle - 1/\sqrt{2}\mid1\rangle$, Alice can perform a rotation of 
$-3\pi/4$ to her qubit. This could be when x = 1 and y = 1.

So basically, Alice will want to rotate by $\pi/4$ when x = 0 and rotate by $3\pi/4$ when x = 1. If y = 1
then this rotation is negative, otherwise it's positive.

\problem{44 b.}

The state of a and b will be as described above, depending on the values of x and y. 

\problem{44 c.}

If we apply a hadamard operation to the state of a and b, we can find x from examining
the vector that is created after the operation. We can find y by taking the negation of x.



\end{document}