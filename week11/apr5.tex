\documentclass[12pt]{article}
 
\usepackage[margin=1in]{geometry} 
\usepackage{amsmath,amsthm,amssymb}
\usepackage{enumitem} 

\newcommand{\problem}[1]{
	\vskip 1em
	{\large \textbf{#1}}
}
 
\begin{document}

\title{CS 1511 Homework 22} % Replace X with the appropriate number
\author{Mathew Varughese, Justin Kramer, Zach Smith} 
\date{Fri, April 5}

\maketitle


\setlength{\parskip}{.2em}
\setlength\parindent{0pt}
 
\problem{43 }

If a $= 0^n$, Simon's algorithm still works. This is because if the function is one-to-one, 
and $a = 0^n$, after we compute $\mid xz\rangle -> \mid x(y \oplus f(x))\rangle$ we 
can measure $\mid (x \oplus a)$ and see that it's equivalent to x. This will let us know
that $a = 0^n$. We will therefore have correctly computed a. Or, if we continue Simon's
algorithm, we will eventually be finding k linear equations for $y \odot a = 0$ with a 
uniform string for y that makes this true. In this case, every single one of these y's will 
work. Solving the linear equations will give us that all values of a are 0, which is true.

\problem{44 a.}
To get the Bell state $1/\sqrt{2}\mid0\rangle + 1/\sqrt{2}\mid1\rangle$, Alice would do 
nothing to her qubit. This will be when x = 0 and y = 0.

\vskip .3cm

To get the Bell state $1/\sqrt{2}\mid0\rangle - 1/\sqrt{2}\mid1\rangle$, Alice can perform a multiplication by
$
\begin{bmatrix} 
	1 & 0 \\
	0 &  -1
\end{bmatrix}
$

 to her qubit. This will be when x = 0 and y = 1.

\vskip .3cm

To get the Bell state $-1/\sqrt{2}\mid0\rangle + 1/\sqrt{2}\mid1\rangle$, Alice can perform a CNOT 
gate operation to her qubit. This will be when x = 1 and y = 0.

\vskip .3cm

To get the Bell state $-1/\sqrt{2}\mid0\rangle - 1/\sqrt{2}\mid1\rangle$, Alice can perform a multiplication by
$
\begin{bmatrix} 
	1 & 0 \\
	0 &  -1
\end{bmatrix}
$
 to her qubit before a CNOT gate operation. This will be when x = 1 and y = 1.

\problem{44 b.}

If x = 0 and y = 0, we get the original state $1/\sqrt{2}\mid00\rangle + 1/\sqrt{2}\mid11\rangle$

If x = 0 and y = 1, we get the state $1/\sqrt{2}\mid00\rangle - 1/\sqrt{2}\mid11\rangle$

If x = 1 and y = 0, we get the state $1/\sqrt{2}\mid10\rangle + 1/\sqrt{2}\mid01\rangle$

If x = 1 and y = 1, we get the state $1/\sqrt{2}\mid01\rangle - 1/\sqrt{2}\mid10\rangle$

\problem{44 c.}

When Bob measures his qubit b, the state of qubit a will collapse to state 0 or 1 with equal probability 1/2
of each occurring. If we then take a CNOT gate to qubits a and b and then a 
Hadamard gate operation on qubit a, we will see the x value as
the "x coordinate" of the resulting vector and the y value as the "y coordinate" of the resulting vector.



\end{document}