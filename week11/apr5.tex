\documentclass[12pt]{article}
 
\usepackage[margin=1in]{geometry} 
\usepackage{amsmath,amsthm,amssymb}
\usepackage{enumitem} 

\newcommand{\problem}[1]{
	\vskip 1em
	{\large \textbf{#1}}
}
 
\begin{document}

\title{CS 1511 Homework 22} % Replace X with the appropriate number
\author{Mathew Varughese, Justin Kramer, Zach Smith} 
\date{Fri, April 5}

\maketitle


\setlength{\parskip}{.2em}
\setlength\parindent{0pt}
 
\problem{43.}

\problem{44 a.}
Alice can perform rotations 
that are multiples of $\pi/4$. Take
the value of x and y combined
and use those to determine
how many degrees to rotate
the qubit by. Say x=0, y=0,
then rotate the qubit by 0 degrees.
x=0, y=1, then rotate by $\pi/4$ degrees.
If x =1 y=0, rotate by $2\pi/4$. If
x = 1, y = 1, rotate by $3\pi/4$ degrees.


\problem{44 b.}
The state of b will be unchanged, 
because Alice performed
rotations only on the first qubit.
The state of a will now be
dependent on the value of x and y.

\problem{44 c.}


\end{document}