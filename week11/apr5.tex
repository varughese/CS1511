\documentclass[12pt]{article}
 
\usepackage[margin=1in]{geometry} 
\usepackage{amsmath,amsthm,amssymb}
\usepackage{enumitem} 

\newcommand{\problem}[1]{
	\vskip 1em
	{\large \textbf{#1}}
}
 
\begin{document}

\title{CS 1511 Homework 22} % Replace X with the appropriate number
\author{Mathew Varughese, Justin Kramer, Zach Smith} 
\date{Fri, April 5}

\maketitle


\setlength{\parskip}{.2em}
\setlength\parindent{0pt}
 
\problem{43.}

If a $= 0^n$, Simon's algorithm still works. This is because if the function is one-to-one, 
and $a = 0^n$, after we compute $\mid xz\rangle -> \mid x(y \oplus f(x))\rangle$ we 
can measure $\mid (x \oplus a)$ and see that it's equivalent to x. This will let us know
that $a = 0^n$. We will therefore have correctly computed a. Or, if we continue Simon's
algorithm, we will eventually be finding k linear equations for $y \odot a = 0$ with a 
uniform string for y that makes this true. In this case, every single one of these y's will 
work. Solving the linear equations will give us that all values of a are 0, which is true.

\problem{44 a.}


\problem{44 b.}

\problem{44 c.}

\end{document}