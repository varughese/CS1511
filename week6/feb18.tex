\documentclass[12pt]{article}
 
\usepackage[margin=1in]{geometry} 
\usepackage{amsmath,amsthm,amssymb, pgfplots}
\usepackage{enumitem} 

\newcommand{\problem}[1]{
	\vskip 1em
	{\large \textbf{#1}}
}
 
\begin{document}

\title{CS 1511 Homework 9} % Replace X with the appropriate number
\author{Mathew Varughese, Justin Kramer, Zach Smith} 
\date{Monday, Feb 18}

\maketitle


\setlength{\parskip}{.2em}
\setlength\parindent{0pt}
 
\problem{16.}
We wish to show that if a Turing Machine $S$ accepts $B$ in LOGSPACE
then there is a Turing Machine $U$ that accepts $A$ in LOGSPACE.

We know that we have a Turing Machine $T$ that takes an input so that
$x \in A \iff T(x) \in B$

The idea of U is to run the output of T
on S and if S accepts, U accepts. Otherwise 
it rejects. However, simulating T 
would take linear space since it's
output takes linear space. It has linear is
output for the following reason.

The output space of T is linearly bounded
by the length of the input ($I$). This is 
because T is LOGSPACE. So, it has a total 
of x maximum positions it can move on the 
tape. (Say length of alphabet is 2, then there
are $log_2{|I|}$ spots, so $2^{log_2{|I|}}$ possibilites,
which simplifies to $|I|$).


We construct U as follows:

U takes in input x. 
U also has a counter variable. 
This counter variable will take log space 
to hold. Run S, and whenever S needs to 
read a value from the output of T store 
the position of T's output it needs 
to read in the counter variable. 
Then rerun T until this space is outputted is generated 
and use that value on S. Repeat this process 
until S is finished. Then U accepts if S accepts and reject 
otherwise.



\problem{17. a)}
A language $C$ is complete for EXPSPACE iff 

1) $C \in EXPSPACE$ and 

2) $\forall L \ \in EXPSPACE, \ \ \ L \leq _r C $

\vskip 1em

Let $C'$ be an arbitrary language in EXPSPACE. 

$ \exists \ TM \ T \ \ \exists k $ such that 

- $\forall I \ T $ halts on $I$ using space $\leq 2^{{|I|}^k}$.

- $\forall I \ T $ accepts $I$ iff $I \in C'$. 

\vskip 1em

First we prove part 2. We have a language $C$.
$C = \{ (I, T, k) \mid T \text{ accepts } I \text{ in } \leq 2^{{|I|}^k} \text{ space } \}$

There exists a turing machine $TM_C$ that decides $C$.

For every $L$ in EXPSPACE, there exists a TM ($T'$) that performs the reduction $L \leq _r C $
in poly time. 

For any language $L$, the Turing Machine $T'$ would return the 
value of $TM_C$ with input $(I, T, k) $.


\end{document}