\documentclass[12pt]{article}
 
\usepackage[margin=1in]{geometry} 
\usepackage{amsmath,amsthm,amssymb}
\usepackage{enumitem} 
\usepackage{listings}

\newcommand{\problem}[1]{
	\vskip 1em
	{\large \textbf{#1}}
}
 
\begin{document}

\title{CS 1511 Homework 25} % Replace X with the appropriate number
\author{Mathew Varughese, Justin Kramer, Zach Smith} 
\date{Friday, Apr 12}

\maketitle


\setlength{\parskip}{.2em}
\setlength\parindent{0pt}
 
\problem{51 a.}
u1 = 0
u2 = 1
u3 = 1

\problem{51 b.}
\begin{lstlisting}
SOLUTION = "011"

def inner_product(a, b):
	a = "{:05b}".format(int(a, 2))
	b = "{:05b}".format(int(b, 2))
	sum = 0
	for a_i, b_i in zip(a, b):
		sum += (int(a_i) * int(b_i)) % 2
	return sum % 2

def outer_product(a, b):
	assert len(a) == len(b)
	product = ""
	for a_i in a:
		for b_i in b:
			product += str((int(a_i) * int(b_i)) % 2)
	return product

def walsh_hadamard(x):
	total = len(x)
	binarys = []
	for i in range(2**total):
		binarys.append("{:05b}".format(i))
	encoded = "".join([str(inner_product(x, b)) for b in binarys])
	return encoded


wh_u = walsh_hadamard(SOLUTION)
uxu = outer_product(SOLUTION, SOLUTION)
wh_uxu = walsh_hadamard(uxu)

print(wh_u, end='')
print(wh_uxu)
\end{lstlisting}


\problem{51 c.}

01100110011001101001100110011001011001101100001111

00001100111100001111001111000000001111111100000000

11110000111111110000000011111111000011111111000000

00000000001111111111111111000000000000000011111111

00000000111111111111111100000000000000001111111111

11111100000000111111111111111100000000000000000000

00000000000011111111111111111111111111111111000000

00000000000000000000000000111111111111111100000000

00000000111111111111111111111111111111110000000000

00000000000000000000001111111111111111111111111111

11110000000000000000

\problem{51 d.}

u1u2 + u2u2 + u3u3 would have 1s in each 
combo of these. To explain further, the 9 bit long 
string 000 000 000 contains digits, and each 
digit corresponds to a combination of the u's. It is
really 
$$ (u_1 u_1) (u_1 u_2) (u_1 u_3) 
(u_2 u_1) (u_2 u_2) (u_2 u_3)
(u_3 u_1) (u_3 u_2) (u_3 u_3) $$

So, $$ 0 1 0
0 1 0
0 0 1 $$ 

is the binary string where each corresponding u term
in the equation specified in the problem has a 1. 
This number in base 10 is 145. 145 + the first 8 bits
used to store the Walsh Hadamard encoding is 153. So look
in this spot.


\problem{52.}
NP = L-PCP(log n)

NP = $\{$L: there is a logspace machine M s.t x $\in$ L iff $\exists$ y : M accepts (x,y) $\}.$

L-PCP(log n) = $\{$L : there is a logspace machine M s.t x $\in$ L iff $\forall$ y : M accepts (x,y) with probability 1 and x $\notin$ L iff $\forall$ y : M rejects (x,y) 
with probability $\geq 1/2 \}$

We need to show two things

\vskip .5 cm

NP $\subseteq$ L-PCP(log n)

L $\in$ NP

$\exists$ M that decides L

This is simple, have the log space verifier tape
of the NP machine M become the random bits
that the L-PCP(log n) uses. 

This will accept and reject with probability 1, which falls
under the L-PCP(log n) conditions.

L $\in$ L-PCP(log n)

\vskip .5 cm

 L-PCP(log n) $\subseteq$ NP

L $\in$ L-PCP(log n)

$\exists$ M that decides L

Run the machine and build a set R that is the random bits used
when the machine accepts for a logirithmic sized R. Then use this set R to build the NP machine
with R as the verifier tape.

L $\in$ NP




\end{document}