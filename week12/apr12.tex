\documentclass[12pt]{article}
 
\usepackage[margin=1in]{geometry} 
\usepackage{amsmath,amsthm,amssymb}
\usepackage{enumitem} 

\newcommand{\problem}[1]{
	\vskip 1em
	{\large \textbf{#1}}
}
 
\begin{document}

\title{CS 1511 Homework 25} % Replace X with the appropriate number
\author{Mathew Varughese, Justin Kramer, Zach Smith} 
\date{Friday, Apr 12}

\maketitle


\setlength{\parskip}{.2em}
\setlength\parindent{0pt}
 
\problem{51 a.}
u1 = 0
u2 = 1
u3 = 1

\problem{52.}
NP = L-PCP(log n)

NP = $\{$L: there is a logspace machine M s.t x $\in$ L iff $\exists$ y : M accepts (x,y) $\}.$

L-PCP(log n) = $\{$L : there is a logspace machine M s.t x $\in$ L iff $\forall$ y : M accepts (x,y) with probability 1 and x $\notin$ L iff $\forall$ y : M rejects (x,y) 
with probability $\geq 1/2 \}$

We need to show two things

\vskip .5 cm

NP $\subseteq$ L-PCP(log n)

L $\in$ NP

$\exists$ M that decides L

This is simple, have the log space verifier tape
of the NP machine M become the random bits
that the L-PCP(log n) uses. 

This will accept and reject with probability 1, which falls
under the L-PCP(log n) conditions.

L $\in$ L-PCP(log n)

\vskip .5 cm

 L-PCP(log n) $\subseteq$ NP

L $\in$ L-PCP(log n)

$\exists$ M that decides L

Run the machine and build a set R that is the random bits used
when the machine accepts for a logirithmic sized R. Then use this set R to build the NP machine
with R as the verifier tape.

L $\in$ NP




\end{document}