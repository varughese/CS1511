\documentclass[12pt]{article}
 
\usepackage[margin=1in]{geometry} 
\usepackage{amsmath,amsthm,amssymb}
\usepackage{enumitem} 

\newcommand{\problem}[1]{
	\vskip 1em
	{\large \textbf{#1}}
}
 
\begin{document}

\title{CS 1511 Homework 24} % Replace X with the appropriate number
\author{Mathew Varughese, Justin Kramer, Zach Smith} 
\date{Wed, Apr 10}

\maketitle


\setlength{\parskip}{.2em}
\setlength\parindent{0pt}
 
\problem{48.}

A language L has a PCP verifier if there's a polynomial-time probablistic algorithm V with
efficiency, completeness, and soundness. 

Call a language L one that has a PCP-verifier using r coing and q adaptive queries.

There is an assumption that the number of queries is at most logarithmic in the input size (n), 
so $2^q$ will still be polynomial to n. 

So our language L with q queries has a logarithmic amount of queries in the input size(n), so 
our non-adaptive proof wil be fine with $2^q$ queries.

\problem{49.}

Let L be the language of pairs $\langle A, k \rangle$ such that A is a 0/1 matrix and k $\in$ Z satisfying perm(A) = k.

To show L $\in$ PCP(poly(n), poly(n)):

In this case, the verifier expects $\pi$ to contain the answer to if each pair $\langle A, k \rangle$ fits perm(A) = k. 

The verifier picks a b $\in \{0, 1\}$ at random and a random A. They then get a value of k, and see if the 
it b matches what they find in $\pi$ for the values of A and k. We can run the following protocol for a polynomial
amount of times to get a result where we can construct a $\pi$ that accepts with probability 1 if perm(A) = k and
accepts at most 1/2 of the time if perm(A) != k.


\problem{50.}

We can run Gaussian Elimination on the matrix
formed by the equations. This operation 
is $n^3$.  Now, taking this system, we
create a system of equations that are similar 
to those in the MAXSAT problem. To rid of the 
rational coefficients, we can simply multiply
by a common factor of all coefficients to
make them whole numbers. We take 
the mod 2 of each coefficient. Then, 
since the equations are linear, we have a 
system very similar to MAXSAT. If the equations
are not satisfiable in this, they will
not be satisfiable be the linear equations.

To find this p, we perform a gap reduction 
just as we did in the MAXSAT problem.


\end{document}