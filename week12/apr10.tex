\documentclass[12pt]{article}
 
\usepackage[margin=1in]{geometry} 
\usepackage{amsmath,amsthm,amssymb}
\usepackage{enumitem} 

\newcommand{\problem}[1]{
	\vskip 1em
	{\large \textbf{#1}}
}
 
\begin{document}

\title{CS 1511 Homework 24} % Replace X with the appropriate number
\author{Mathew Varughese, Justin Kramer, Zach Smith} 
\date{Wed, Apr 10}

\maketitle


\setlength{\parskip}{.2em}
\setlength\parindent{0pt}
 
\problem{48.}
Problem 11.2

% not sure
If there is a verifier that can do this,
then there is a verifier that can
try every possible query. 

\problem{49.}
Problem 11.7
Consider the following problem: Given a system of linear equations in n with coeffi- cients that are rational numbers, determine the largest subset of equations that are simultaneously satisfiable. Show that there is a constant ρ < 1 such that approximat- ing the size of this subset is NP-hard.


\problem{50.}
Problem 11.16

We can run Gaussian Elimination on the matrix
formed by the equations. This operation 
is $n^3$.  Now, taking this system, we
create a system of equations that are similar 
to those in the MAXSAT problem. To rid of the 
rational coefficients, we can simply multiply
by a common factor of all coefficients to
make them whole numbers. We take 
the mod 2 of each coefficient. Then, 
since the equations are linear, we have a 
system very similar to MAXSAT. If the equations
are not satisfiable in this, they will
not be satisfiable be the linear equations.

To find this $p$, we perform a gap reduction 
just as we did in the MAXSAT problem.


\end{document}