\documentclass[12pt]{article}
 
\usepackage[margin=1in]{geometry} 
\usepackage{amsmath,amsthm,amssymb}
\usepackage{enumitem} 

\newcommand{\problem}[1]{
	\vskip 1em
	{\large \textbf{#1}}
}
 
\begin{document}

\title{CS 1511 Homework 23} % Replace X with the appropriate number
\author{Mathew Varughese, Justin Kramer, Zach Smith} 
\date{Monday, Apr 8}

\maketitle


\setlength{\parskip}{.2em}
\setlength\parindent{0pt}
 
\problem{45.}
Extra credit ...


\problem{46.}
BQP is the complexity class that
contains languages that are 
solvable by a quantum computer 
in polynomial time with an error 
probability of 1/3.

Add another qubit to the register. When
the qubit is zero, all amplitudes
correspond to the real part of the
amplitudes in the original algorithm. When
it is one, the amplitudes correspond to
the imaginary part of the amplitudes of
the original algorithm.

This means that the states can be either
the real part or imaginary part
of the matricies. The state of them
depends of the state of the qubit. This is
because of the linearity of the matrix operations.
The real or imaginary part can be chosen
based on the qubit.


\end{document}