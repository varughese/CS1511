\documentclass[12pt]{article}
 
\usepackage[margin=1in]{geometry} 
\usepackage{amsmath,amsthm,amssymb}
\usepackage{enumitem} 

\newcommand{\problem}[1]{
	\vskip 1em
	{\large \textbf{#1}}
}
 
\begin{document}

\title{CS 1511 Homework 23} % Replace X with the appropriate number
\author{Mathew Varughese, Justin Kramer, Zach Smith} 
\date{Monday, Apr 8}

\maketitle


\setlength{\parskip}{.2em}
\setlength\parindent{0pt}
 
\problem{45.}
Showing that final state of qubit b is the initial state of qubit x.

When Alice measures, qubit b collapses to one of four possible states.
As shown in the previous problem, there are four ways to take the final
state of qubit b and find the initial state x. According to our last homework,
the values of the two classical bits can decide what operation to proceed with.

For example, in one case (x = 0 and y = 0) that qubit x was not modified at
all, so the initial and final state of x = b.

If (x = 0 and y = 1) we could take qubit b, apply a Hadamard operation and multiply
by 
$
\begin{bmatrix} 
	1 & 0 \\
	0 &  -1
\end{bmatrix}
$
to get the initial state of x.

If (x = 1 and y = 0), we could take qubit b and apply a CNOT operation to it to 
get qubit x.

If (x = 1 and y =1),  we could take qubit b and apply a CNOT operation to it before
multiplying by
$
\begin{bmatrix} 
	1 & 0 \\
	0 &  -1
\end{bmatrix}
$
to get qubit x at its initial state.

This is possible because all the operations are reversible in quantum.

After negation of x,

a = 1

b = $\mid0\rangle/\sqrt{2} +  \mid1\rangle/\sqrt{2}$

x = $\alpha\mid1\rangle +  \beta\mid0\rangle$

After Hadamard gate,

a = $\mid0\rangle/\sqrt{2} +  \mid1\rangle/\sqrt{2}$

b = $\mid0\rangle/\sqrt{2} +  \mid1\rangle/\sqrt{2}$

x = $\alpha\mid1\rangle +  \beta\mid0\rangle$

After measuring a and x

a = what's measured

b = $\mid0\rangle/\sqrt{2} +  \mid1\rangle/\sqrt{2}$

x = what's measured

Bob can run b through a Hadamard gate and then negate x if
a = 1 to get the original state of x.


\problem{46.}
BQP is the complexity class that
contains languages that are 
solvable by a quantum computer 
in polynomial time with an error 
probability of 1/3.

Add another qubit to the register. When
the qubit is zero, all amplitudes
correspond to the real part of the
amplitudes in the original algorithm. When
it is one, the amplitudes correspond to
the imaginary part of the amplitudes of
the original algorithm.

This means that the states can be either
the real part or imaginary part
of the matricies. The state of them
depends of the state of the qubit. This is
because of the linearity of the matrix operations.
The real or imaginary part can be chosen
based on the qubit.


\end{document}