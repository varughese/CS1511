\documentclass[12pt]{article}
 
\usepackage[margin=1in]{geometry} 
\usepackage{amsmath,amsthm,amssymb}
\usepackage{enumitem}

\newcommand{\problem}[1]{
	\vskip 1em
	{\large \textbf{#1}}
}
 
\begin{document}

\title{CS 1511 Homework 12} % Replace X with the appropriate number
\author{Mathew Varughese, Justin Kramer, Zach Smith} 
\date{Wednesday, Feb 27}

\maketitle


\setlength{\parskip}{.2em}
\setlength\parindent{0pt}
 
\problem{21. part b}

EXACT INDSET = \{$<G, k> \mid $ the largest independent set in G has size exactly k\}.

To show EXACT INDSET $\in$ $\pi_2^p$, we need to demonstrate two things.

$\forall x \forall y \forall z$ T(x, y, z) runs in time $\mid x \mid^k$

$\forall x \exists y$ T(x, y, z) accepts iff x $\in$ EXACT INDSET

Our language will now become $\forall x$, with x being an independent set
of G, $\exists w$ where w is the largest independent set of size exactly k.

To solve this in poly-time, we will need 3 read-only tapes and 1 work tape, 
which is based on a simple scaling from our 2 read-only 1 work tape model
with $\pi_1^p$. 

Basically, it reads on the second tape the set of vertices. It goes through the tape 
and checks each vertex in the set and marks it on the graph (on the first tape). 
When it does this, 
it makes sure adjacent vertices are not touching to ensure it is an independent
set. 

Then, it checks the third tape, which is another set. It just needs to validate 
that this set is also a indepdent set and that it is larger or equal to in 
size to the set on the 2nd tape.

\problem{21. part c}



\end{document}