\documentclass[12pt]{article}
 
\usepackage[margin=1in]{geometry} 
\usepackage{amsmath,amsthm,amssymb}
\usepackage{enumitem}

\newcommand{\problem}[1]{
	\vskip 1em
	{\large \textbf{#1}}
}
 
\begin{document}

\title{CS 1511 Homework 16} % Replace X with the appropriate number
\author{Mathew Varughese, Justin Kramer, Zach Smith} 
\date{Monday, March 18}

\maketitle


\setlength{\parskip}{.2em}
\setlength\parindent{0pt}
 

\problem{30.)}

Let L $\in$ BPL.

$\exists $ TM T and $\exists$ integer k such that:

$\forall x \forall R$ T(x,R) halts using log space

if x $\in$ L then prob(T(x,R) accepts) $\geq$ 3/4

if x $\notin$ L then prob(T(x,R) accepts) $\leq$ 1/4

\vskip 0.3cm

LOGSPACE $ \subset $ P. This is because with $\log{n}$ space, 
there are $2^{\log{n}}$ possible configurations for R. 
This simplifies to polynomial time. 

\vskip 0.3cm

Now, there exists some $u$, where $u$ is a random assignment of 

Take an input x that has a length of n. Now we will have C be the number

of configurations of TM T with input x. We will combine each configuration

with another set of the same configurations to form a matrix. We will create 

this matrix such that each cell will have a 1/2 probability if the second configuration

is reachable from the first in one step, and a probability of 0 otherwise. With

this created, each cell $W_t$ with configurations $c_1$ and $c_2$ is the probability

of reaching configuration $c_2$ from $c_1$ in t steps, where $W_t$ is the matrix

created by multiplying W by itself t times. By scaling this up, we can compute the

accepting probability of T(x, R) and decide if x $\in$ L. With this we can see that each 

probability is a multiple of $1/2^{poly(n)}$, which we can represent with a poly

number of digits. Therefore, L $\in$ P and BPL $\subseteq$ P.

\problem{31.)}

If $L \in \text{BP} \cdot \text{NP}$, this means that we can probabliscally
reduce $\overline{3SAT}$ to 3SAT. So, for most reductions, this 
reduction will work. We wish to change this to be a $\exists \forall \exists $
problem. 

We already proved that BP $ \cdot $ NP $\subseteq NP/Poly$. Therefore
we know since co3SAT is in this language there is a family of circuits ($\exists
C_n$) that decides 3SAT nondeterministically.

\vskip 0.3cm

From here, if we can show by repeatedly performing pairings of random
strings, we can increase the probability to show that 

If the length of the random string R is $m$, we can say there are
$2^m$ different possible $m$'s. The probability that $T$ is incorrect
is less than $\frac{1}{4^n}$ where $n$ is the length of $x$.

\vskip 0.3cm

Now in the BPP $\subset \Sigma_{2}^{p}$ proof we married (pair) each R with another R to decrease the expected
number of R's that cause the machine to error. When we perform
k marriages, the expected number of $R$s that cause the machine
to error will be less than 1. This occurs when $k = \frac{m}{2n}$. Here we do
something similar. We perform k marriages of reductions so that the number of
"incorrect" reductions becomes less than 1. Then there must exist some 
random pairing of reductions such that all reductions work. 

\vskip 0.3cm

Again, since we assuming that the complement of 3SAT $\in$ NP/poly, we can say that
there exists a poly-sized circuit family that can decide the complement of 3 SAT with 
a poly-sized string y that verifies if x $\in$ L. 

In other words, we have $\exists C_n\forall u \exists r$ and
$C_n$ is a circuit family that reduces if $\overline{3SAT}$ to 3SAT
with either input $C_n$(r, u) or $C_n$(r, $r \oplus u$). 

Since $\overline{3SAT}$ can be reduced to 3SAT by a $\Sigma_3^p$ algorithm, 
this means  $\overline{3SAT}$ would be complete for $\Pi_3^p$. This means 
that 
$\Pi_3^p \subseteq \Sigma_3^p$, which would make PH collapse to $\Sigma_3^p$.



\end{document}
