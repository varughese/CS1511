\documentclass[12pt]{article}
 
\usepackage[margin=1in]{geometry} 
\usepackage{amsmath,amsthm,amssymb}
\usepackage{enumitem}

\newcommand{\problem}[1]{
	\vskip 1em
	{\large \textbf{#1}}
}
 
\begin{document}

\title{CS 1511 Homework 17} % Replace X with the appropriate number
\author{Mathew Varughese, Justin Kramer, Zach Smith} 
\date{Friday, March 22}

\maketitle


\setlength{\parskip}{.2em}
\setlength\parindent{0pt}
 

\problem{32.)}

The language QNR = \{ (a, p) $\mid$ a is not a quadratic residue modulo p where p is a prime.\}

\vskip 0.3cm

From Euler's criterion, we know that there are (p+1)/2 quadratic residues and (p-1)/2 quadratic
nonresidues.

\vskip 0.3cm

In the private coin protocol, the verifier takes a random number r mod p and a random bit b $\in$ \{0,1\}
and sends the prover $r^2$ mod p if b $=$ 0 and sends the prover $ar^2$ mod p if b $=$ 1.

\vskip 0.3cm

If a is a quadratic residue, then the prover has a 1/2 chance of guessing b correctly. Otherwise,
the prover is certain of the value of b.

\vskip 0.3cm

We know from Euler's criterion that if a is not a quadratic residue, then $\mid S\mid$ will be equal to 
(p-1)/2. If it is a quadratic residue, $\mid S\mid$ will be equal to (p+1)/2. With the set S being the
set of quadratic or non-quadratic residues.

\vskip 0.3cm

To solve this problem, we can use several iterations of the set lower bound protocol run in parallel.
In this instance of the protocol, we will have both our prover and verifier know a number K. The prover
will try to convince the verifier that $\mid S\mid$ is $\geq$ K and the verifier should reject with good 
probability if $\mid S\mid \leq K/2$. Let k be an integer such that $2^{k-2} < K \leq 2^{k-1}$.

\vskip 0.3cm

Verifier: Randomly pick a function h : $\{0,1\}^m -> \{0,1\}^k$ from a pairwise independent hash function collection 
$H_{m,k}$. Pick y $\in_R \{0,1\}^k$. Send h,y to prover.

\vskip 0.3cm

Prover: Try to find an x $\in$ S such that H(x) $=$ y. Send such an x to Verifier, together with a certificate that
x $\in$ S.

\vskip 0.3cm

Verifier output: If h(x) $=$ y and the certificate validates that x $\in$ S then accept, otherwise reject.

\vskip 0.3cm

If run for several iterations where the verifier accepts iff the traction of accepting iterations was least 5/16,
which is the mean of 3/8 and 1/4. The Chernoff bound shows that a constant number of iterations will ensure completeness
probability of at least 2/3. These iterations are run in parallel by sending several choices of h,y at once to the prover.

\vskip 0.3cm

This is the public coin interative protocol for the language QNR.



\end{document}
