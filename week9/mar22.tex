\documentclass[12pt]{article}
 
\usepackage[margin=1in]{geometry} 
\usepackage{amsmath,amsthm,amssymb}
\usepackage{enumitem}

\newcommand{\problem}[1]{
	\vskip 1em
	{\large \textbf{#1}}
}
 
\begin{document}

\title{CS 1511 Homework 17} % Replace X with the appropriate number
\author{Mathew Varughese, Justin Kramer, Zach Smith} 
\date{Friday, March 22}

\maketitle


\setlength{\parskip}{.2em}
\setlength\parindent{0pt}
 

\problem{32.)}

Let L $\in$ BPL.

$\exists $ TM T and $\exists$ integer k such that:

$\forall x \forall R$ T(x,R) halts using log space

if x $\in$ L then prob(T(x,R) accepts) $\geq$ 3/4

if x $\notin$ L then prob(T(x,R) accepts) $\leq$ 1/4

\vskip 0.3cm

LOGSPACE $ \subset $ P. This is because with $\log{n}$ space, 
there are $2^{\log{n}}$ possible configurations for R. 
This simplifies to polynomial time. 

\vskip 0.3cm

Now, there exists some $u$, where $u$ is a random assignment of 

Take an input x that has a length of n. Now we will have C be the number

of configurations of TM T with input x. We will combine each configuration

with another set of the same configurations to form a matrix. We will create 

this matrix such that each cell will have a 1/2 probability if the second configuration

is reachable from the first in one step, and a probability of 0 otherwise. With

this created, each cell $W_t$ with configurations $c_1$ and $c_2$ is the probability

of reaching configuration $c_2$ from $c_1$ in t steps, where $W_t$ is the matrix

created by multiplying W by itself t times. By scaling this up, we can compute the

accepting probability of T(x, R) and decide if x $\in$ L. With this we can see that each 

probability is a multiple of $1/2^{poly(n)}$, which we can represent with a poly

number of digits. Therefore, L $\in$ P and BPL $\subseteq$ P.


\end{document}
