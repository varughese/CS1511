\documentclass[12pt]{article}
 
\usepackage[margin=1in]{geometry} 
\usepackage{amsmath,amsthm,amssymb}
\usepackage{enumitem}

\newcommand{\problem}[1]{
	\vskip 1em
	{\large \textbf{#1}}
}
 
\begin{document}

\title{CS 1511 Homework 17} % Replace X with the appropriate number
\author{Mathew Varughese, Justin Kramer, Zach Smith} 
\date{Friday, March 22}

\maketitle


\setlength{\parskip}{.2em}
\setlength\parindent{0pt}
 

\problem{32.)}

The language QNR = \{ (a, p) $\mid$ a is not a quadratic residue modulo p where p is a prime.\}

\vskip 0.3cm

From Euler's criterion, we know that there are (p+1)/2 quadratic residues and (p-1)/2 quadratic
nonresidues.

\vskip 0.3cm

In the private coin protocol, the verifier takes a random number r mod p and a random bit b $\in$ \{0,1\}
and sends the prover $r^2$ mod p if b $=$ 0 and sends the prover $ar^2$ mod p if b $=$ 1.

\vskip 0.3cm

If a is a quadratic residue, then the prover has a 1/2 chance of guessing b correctly. Otherwise,
the prover is certain of the value of b.

\vskip 0.3cm

We know from Euler's criterion that if a is not a quadratic residue, then $\mid S\mid$ will be equal to 
(p-1)/2. If it is a quadratic residue, $\mid S\mid$ will be equal to (p+1)/2. With the set S being the
set of quadratic or non-quadratic residues.



\end{document}
